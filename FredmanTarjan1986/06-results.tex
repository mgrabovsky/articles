F-heaps have several additional applications. H.~Gabow (private communication, Oct.
1984) has noted that they can be used to speed up the scaling algorithm of Edmonds
and Karp~\cite{EdmondsKarp1972} for minimum-cost network flow from \bigO{m^2
(\log_{(m / n + 2)} n) (\log N)} to \bigO{m (n \log n + m) (\log N)}, where \(N\) is
the maximum capacity, assuming integer capacities. They can also be used to find
shortest pairs of disjoint paths in \bigO{n \log n + m} time~%
\cite{GabowStallman1985}, improved from \bigO{m \log_{(m / n + 2)} n}~%
\cite{SuurballeTarjan1984}.

A less immediate application is to the directed analogue of the minimum spanning tree
problem -- the optimum branching problem. Gabow, Galil, and Spencer~\cite{Gabow+1984}
have proposed a very complicated \bigO{n \log n + m \log \log \log_{(m / n + 2)}
n}-time algorithm for this problem, improving on Tarjan's~\cite{Camerini+1979,
Tarjan1977} bound of \bigO{\min \{ m \log n, n^2 \}}. F-heaps can be used to solve
this problem in \bigO{n \log n + m} time~\cite{Gabow+1986}.

Another recent major result is an improvement by Gabow, Galil, and Spencer~%
\cite{Gabow+1984} to our minimum spanning tree algorithm of
\autoref{sec:shortest-paths}. They have improved our time bound from \bigO{m \beta(m,
n)} to \bigO{m \log \beta(m, n)} by introducing the idea of grouping edges with
a common vertex into ``packets'' and working only with packet minima.
(See~\cite{Gabow+1986}.)

Several intriguing open questions are raised by our work:
\begin{enumerate}[(i)][3]
	\item Is there a ``self-adjusting'' version of F-heaps that achieves the same
		amortized time bounds, but requires neither maintaining ranks nor performing
		cascading cuts? The self-adjusting version of leftist heaps proposed by
		Sleator and Tarjan~\cite{SleatorTarjan1986} does not solve this problem, as
		the amortized time bound for \oper{DecreaseKey} is \bigO{\log n} rather than
		\bigO{1}. We have a candidate data structure, but are so far unable to verify
		that it has the desired time bounds.
	\item Can the best time bounds for finding shortest paths and minimum spanning
		trees be improved? Dijkstra's algorithm requries \(\Omega(n \log n + m)\)
		time assuming a comparison-based computation model, since it mus texamine all
		the edges in the worst case and can be used to sort \(n\) numbers.
		Nevertheless, this does not preclude the existence of another, faster
		algorithm. Similarly, there is no reason to suppose tha the
		Gabow--Galil--Spencer bound for minimum spanning trees is the best possible.
		It is suggestive that the minimality of a spanning tree can be checked in
		\bigO{m \alpha(m, n)} time~\cite{Tarjan1979b} and even in \bigO{m}
		comparison~\cite{Komlos1984}. Furthermore, if the edges are presorted,
		a minimum spanning tree can be computed in \bigO{m \alpha(m, n)} time~%
		\cite{Tarjan1983}.
	\item Are there other problems needing heaps where the use of F-heaps gives
		aymptotically improved algorithms? A possible candidate is the nonbipartite
		weighted matching problem, for which the current best bound of \bigO{n^2 \log
		n + n m \log \log \log_{(m / n + 2)} n}~\cite{Gabow+1984} might be improvable
		to \bigO{n^2 \log n + n m}.
\end{enumerate}

