In this section we use F-heaps to implement Dijkstra's shortest path algorithm~%
\cite{Dijkstra1959} and explore some of the consequences. Our discussion of
Dijkstra's algorithm is based on Tarjan's presentation~\cite{Tarjan1983}. Let \(G\)
be a directed graph, one of whose vertices is distinguished as the \term{source}
\(s\), and each of whose edges \((v, w)\) has a nonnegative \term{length} \(l(v,
w)\). The \term{length} of a path in \(G\) is the sum of its edge lengths; the
\term{distance} from a vertex \(v\) to a vertex \(w\) is the minimum length of a path
from \(v\) to \(w\). A minimum-length path from \(v\) to \(w\) is called
a \term{shortest path}. The \term{single-source shortest path problem} is that of
finding a shortest path from \(s\) to \(v\) for every vertex \(v\) in \(G\).

We shall denote the number of vertices in \(G\) by \(n\) and the number of edges by
\(m\). We assume that there is a path from \(s\) to any other vertex; thus \(m \geq
n - 1\). Dijkstra's algorithm solves the shortest path problem using
a \term{tentative distance function} \(d\) from vertices to real numbers with the
following properties:
\begin{enumerate}
    \item For any vertex \(v\) such that \(d(v)\) is finite, there is a path from
        \(s\) to \(v\) of length \(d(v)\);
    \item when the algorithm terminates, \(d(v)\) is the distance from \(s\) to
    \(v\).
\end{enumerate}

Initially \(d(s) = 0\) and \(d(v) = \infty\) for \(v \neq s\). During the running of the algorithm, each vertex is one of three states: \term{unlabeled}, \term{labeled}, or \term{scanned}. Initially \(s\) is labeled and all other vertices are unlabeled. The algorithm consists of repeating the following step until there are no labeled vertices (every vertex is scanned).

\begin{step}{Scan}
    Select a labeled vertex \(v\) with \(d(v)\) minimum. Convert \(v\) from labeled
    to scanned. For each edge \((v, w)\) such that \(d(v) + l(v, w) < d(w)\), replace
    \(d(w)\) by \(d(v) + l(v, w)\) and make \(w\) labeled.
\end{step}

The nonnegativity of the edge lengths implies that a vertex, once scanned, can never
become labeled, and further that the algorithm computes distances correctly.

To implement Dijkstra's algorithm, we use a heap to store the set of labeled
vertices. The tentative distance of a vertex is its key. Initialization requires one
\oper{MakeHeap} and one \oper{Insert} operation. Each scanning step requires one
\oper{DeleteMin} operation. In addition, for each edge \((v, w)\) such that \(d(v)
+ l(v, w) < d(w)\), we need either an \oper{Insert} operation (if \(d(w) = \infty\))
or a \oper{DecreaseKey} operation (if \(d(w) < \infty\)). Thus there is one
\oper{MakeHeap} operation, \(n\) \oper{Insert} and \(n\) \oper{DeleteMin} operations,
and at most \(m\) \oper{DecreaseKey} operations. The maximum heap size is \(n - 1\).
If we use an F-heap, the total time for heap operations is \bigO{n \log n + m}. The
time for other tasks is \bigO{n + m}. Thus we obtain an \bigO{n \log n + m } running
time for Dijkstra's algorithm. This improves Johnson's bound of \bigO{m \log_{(m/n
+ 2)} n}~\cite{Johnson1977,Tarjan1983} based on the use of implicit heaps with
a branching factor of \(m/n + 2\).

By augmenting the algorithm, we can make it compute shortest paths as well as
distances: For each vertex \(v\), we maintain a \term{tentative predecessor}
\(pred(v)\) that immediately precedes \(v\) on a tentative shortest path from \(s\)
to \(v\). When replacing \(d(w)\) by \(d(v) + l(v, w)\) in a scanning step, we define
\(pred(w) = v\). Once the algorithm terminates, we can find a shortest path from
\(s\) to \(v\) for any vertex \(v\) by following predecessor pointers from \(v\) back
to \(s\). Augmenting the algorithm to maintain predecessors adds only \bigO{m} to the
running time.

Several other optimization algorithms use Dijkstra's algorithm as a subroutine, and
for each of these, we obtain a corresponding improvement. We shall give three
examples:
\begin{enumerate}
    \item The all-pairs shortest path problem, with or without negative edge lengths,
        can be solved in \bigO{nm} time plus \(n\) iterations of Dijkstra's
        algorithm~\cite{Johnson1977,Tarjan1983}. Thus we obtain a time bound of
        \bigO{n^2 \log n + n m}, improved from \bigO{n m \log_{(m/n + 2)} n}.
    \item The assignment problem (bipartite weighted matching) can also be solved in
        \bigO{nm} time plus \(n\) iterations of Dijkstra's algorithm~%
        \cite{Tarjan1983}.  Thus we obtain a bound of \bigO{n^2 \log n + n m},
        improved from \bigO{n m \log_{(m/n + 2)} n}.
    \item Knuth's generalization~\cite{Knuth1977} of Dijkstra's algorithm to compute
        minimum-cost derivations in a context-free language runs in \bigO{n \log
        n + t} time, improved from \bigO{m \log n + t}, where \(n\) is the number of
        nonterminals, \(m\) is the number of productions, and \(t\) is the total
        length of the productions. (There is at least one production per nonterminal;
        thus \(m \geq n\).)
\end{enumerate}

% vim: set et sw=4:
