A \term{heap} is an abstract data structure consisting of a set of \term{items}, each
with a real-valued \term{key}, subject to the following operations:
\begin{itemize}
    \item \oper{MakeHeap}: Return a new, empty heap.
    \item \opargs{Insert}{i, h}: Insert a new item \(i\) with predefined key into heap
        \(h\).
    \item \opargs{FindMin}{h}: Return an item of minimum key in heap \(h\). This
        operation does not change \(h\).
    \item \opargs{DeleteMin}{h}: Delete an item of minimum key from heap \(h\) and
        return it.
\end{itemize}

In addition, the following operations on heaps are often useful:
\begin{itemize}
    \item \opargs{Meld}{h_1, h_2}: Return the heap formed by taking the union of the
        item-disjoint heaps \(h_1\) and \(h_2\). This operation destroys \(h_1\) and
        \(h_2\).
    \item \opargs{DecreaseKey}{\Delta, i, h}: Decrease the key of item \(i\) in heap
        \(h\) by subtracting the nonnegative real number \(\Delta\). This operation
        assumes that the position of \(i\) in \(h\) is known.
    \item \opargs{Delete}{i, h}: Delete arbitrary item \(i\) from heap \(h\). This
        operation assumes that the positino of \(i\) in \(h\) is known.
\end{itemize}

Note that other authors have used different terminology for heaps.  Knuth~%
\cite{TAOCP1} called heaps ``priority queues.'' Aho et al.~\cite{Aho+1974} used
this term for heaps not subject to melding and called heaps subject to melding
``mergeable heaps.''

In our discussion of heaps, we assume that a given item is in only on heap at a time
and that a pointer to its heap position is maintained. It is important to remember
that heaps do \emph{not} support efficient searching for an item.

\begin{remark}
    The heap parameter \(h\) in \oper{DecreaseKey} and \oper{Delete} is actually
    redundant, since item \(i\) determines \(h\). However, our implementations of
    these operations require direct access to \(h\), which must be provided by an
    auxiliary data structure if \(h\) is not a parameter to the operation. If melding
    is allowed, a data structure for disjoint set union~\cite{TarjanVanLeeuwen1984}
    must be used for this purpose; otherwise, a pointer from each item to the heap
    containing it suffices.
\end{remark}

Vuillemin~\cite{Vuillemin1978} invented a class of heaps, called \term{binomial
queues}, that support all the heap operations in \bigO{\log n} worst-case time. Here
\(n\) is the number of items in the heap or heaps involved in the operation.
Brown~\cite{Brown1978} studied alternative representations of binomial heaps and
developed both theoretical and experimental running-time bounds. His results suggest
that binomial queues are a good choice in practice if meldable heaps are needed,
although several other heap implementations have the same \bigO{\log n} worst-case
time bound. For further discussion of heaps, see~\cite{Aho+1974,Brown1978,TAOCP3,Tarjan1983}.

In this paper we develop an extension of binomial queues called \term{Fibonacci
heaps}, abbreviated \term{F-heaps}. F-heaps support \oper{DeleteMin} and
\oper{Delete} in \bigO{\log n} amortized time, and all the other heap operations, in
particular \oper{DecreaseKey}, in \bigO{1} amortized time. For situations in which
the number of deletions is small compared to the total number of operations, F-heaps
are asymptotically faster than binomial queues.

Heaps have a variety of applications in network optimization, and in many such
applications, the number of deletions is relatively small. Thus we are able to use
F-heaps to obtain asymptotically faster algorithms for several well-known network
optimization problems. Our original purpose in developing F-heaps was to speed up
Dijkstra's algorithm for the single-source shortest path problem with non-negative
length edges~\cite{Dijkstra1959}. Our implementation of Dijkstra's algorithm runs in
\bigO{n \log n + m} time, improved from Johnson's \bigO{m \log_{(m/n + 2)} n} bound~%
\cite{Johnson1977,Tarjan1983}.

Various other network algorithms use Dijkstra's algorithm as a subroutine, and for
each of these, we obtain a corresponding improvement. Thus we can solve both the
all-pairs shortest path problem with possibly negative length edges and the
assignment problem (bipartite weighted matching) in \bigO{n^2 \log n + n m} time,
improved from \bigO{n m \log_{(m/n + 2)} n}~\cite{Tarjan1983}.

We also obtain a faster method for computing minimum spanning trees. Our bound is
\bigO{m \beta(m, n)}, improved from \bigO{m \log \log_{(m/n + 2)} n}~%
\cite{CheritonTarjan1976,Tarjan1983}, where \[\beta(m, n) = \min \{ i \mid \log^{(i)}
n \leq m/n \}.\]

All our bounds for network optimization are asymptotic improvements for graphs of
intermediate density (\(n \ll m \ll n^2\)). Our bound for minimum spanning trees,
which is perhaps our most striking result, is an asymptotic improvement for sparse
graphs as well.

The remainder of the paper consists of five sections. In \autoref{sec:f-heaps} we
develop and analyze F-heaps. In \autoref{sec:variants} we discuss variants of F-heaps
and some additional heap operations. In \autoref{sec:shortest-paths} we use F-heaps
to implement Dijkstra's algorithm. In \autoref{sec:mst} we discuss the minimum
spanning tree problem. In \autoref{sec:results} we mention several more recent
results and remaining open problems.

% vim: set et sw=4:
