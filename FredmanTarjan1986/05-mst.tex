A less immediate application of F-heaps is to the computation of minimum spanning
trees. See~\cite{Tarjan1983} for a systematic discussion of minimum spanning tree
algorithms. Let \(G\) be a connected undirected graph with \(n\) vertices and \(m\)
edges \((v, w)\) each of which has a nonnegative \term{cost} \(c(v, w)\).
A \term{minimum spanning tree} of \(G\) is a spanning tree of \(G\) of minimum total
edge cost.

We can find a minimum spanning tree by using a generalized greedy approach. We
maintain a forest defined by the edges so far selected to be in the minimum spanning
tree. We initialize the forest to contain each of the \(n\) vertices of \(G\) as
a one-vertex tree. We initialize the forest to contain each of the \(n\) vertices of
\(G\) as a one-vertex tree. Then we repeat the following step \(n - 1\) times (until
there is only one \(n\)-vertex tree).

\begin{step}{Connect}
	Select any tree \(T\) in the forest. Find a minimum-cost edge with exactly one
	endpoint in \(T\) and add it to the forest. (This connects two trees to form
	one.)
\end{step}

This algorithm is nondeterministic: We are free to select the tree to be processed in
each connecting step in any way we wish. One possibility is to always select the tree
containing a fixed start vertex \(s\), thereby growing a single tree \(T\) that
absorbs the vertices of \(G\) one by one. This algorithm, usually credited to Prim~%
\cite{Prim1957} and Dijkstra~\cite{Dijkstra1959} independently, was actually
discovered by Jarník~\cite{Jarnik1930} (see Graham and Hell's survey paper~%
\cite{GrahamHell1985}). The algorithm is almost identical to Dijkstra's shortest path
algorithm, and we can implement it in the same way. For each vertex \(v\) not yet in
\(T\), we maintain a key measuring the tentative cost of connecting \(v\) to \(T\).
If \(v \neq s\) and \(key(v) < \infty\), we also maintain an edge \(e(v)\) by which
the connection can be made. We start by defining \(key(s) = 0\) and key \(key(v)
= \infty\) for \(v \neq s\). Then we repeat the following step until no vertex has
finite key:

\begin{step}{Connect to start}
	Select a vertex \(v\) with \(key(v)\) minimum among vertices with finite key.
	Replace \(key(v)\) by \(-\infty\). For each edge \((v, w)\) such that \(c(v, w)
	< key(w)\), replace \(key(w)\) by \(c(v, w)\) and define \(e(w) = (v, w)\).
\end{step}

When this algorithm terminates, the set of edges \(e(v)\) with \(v \neq s\) defines
a minimum spanning tree. The purpose of setting \(key(v) = -\infty\) in the
connecting step is to mark \(v\) as being in \(T\). If we store the vertices with
finite key in an F-heap, the algorithm requires \(n\) \oper{DeleteMin} operations and
\bigO{m} other heap operations, none of them deletions. The total running time is
\bigO{n \log n + m}.

The best previous bound for computing minimum spanning trees is \bigO{m \log
\log_{(m/n + 2)} n}~\cite{CheritonTarjan1976,Tarjan1983}, a slight improvement over
Yao's \bigO{m \log \log n} bound~\cite{Yao1975}. Our \bigO{n \log n + m} bound is
better than the old bound for graphs of intermediate density (e.g., \(m = \Theta(n
\log n)\)). However, we can do better.

The idea is to grow a single tree only until its heap of neighboring vertices exceeds
a certain critical size. Then we start from a new vertex and grow another tree, again
stopping when the heap gets too large. We continue in this way until every vertex is
in a tree. Then we condense every tree into a single supervertex and begin a new pass
of the same kind over the condensed graph. After a sufficient number of passes, only
one supervertex will remain, and by expanding the supervertices, we can extract
a minimum spanning tree.

Our implementation of this method does the condensing implicitly. We shall describe
a single pass of the algorithm. We begin with a forest of previously grown trees,
which we call \term{old trees}, defined by the edges so far added to the forest. The
pass connects these old trees into new larger trees that become the old trees for the
next pass.

To start the pass, we number the old trees consecutively from one and assign to each
vertex the number of the tree containing it. Thus allows us to refer to trees by
number and to directly access the old tree \(tree(v)\) containing any vertex \(v\).
Next, we do a \term{cleanup}, which takes the place of condensing. We discard every
edge connecting two vertices in the same old tree and all but a minimum-cost edge
connecting each pair of old trees. We can do such a cleanup in \bigO{m} time by
sorting the edges lexicographically on the number of the trees containing their
endpoints, using a two-pass radix sort. Then we scan the sorted edge list, saving
only the appropriate edges. After the cleanup we construct a list for each old tree
\(T\) of the edges with one endpoint in \(T\). (Each edge will appear in two such
lists.)

To grow new trees, we give every old tree a key of \(\infty\) and unmark it. We
create an empty heap. Then we repeat the following tree-growing step until there are
no unmarked old trees. This completes the pass.

\begin{step}{Grow a new tree}
	Select any unmarked old tree \(T_0\), and insert it as an item into th heap with
	a key of zero. Repeat the following step until the heap is empty, or it contains
	more than \(k\) trees (where \(k\) is a parameter to be chosen later), or the
	growing tree becomes connected to a marked old tree:

	\begin{step}{Connect to starting tree}
		Delete an old \(T\) of minimum key from the heap. Set \(key(T) = -\infty\).
		If \(T \neq T_0\) (i.e., \(T\) is not the starting tree of this growth step),
		add \(e(T)\) to the forest. (Edge \(e(T)\) connects old tree \(T\) to the
		current tree containing \(T_0\).) If \(T\) is marked, stop growing the
		current tree, and finish the growth step as described below. Otherwise, mark
		\(T\). For each edge \((v, w)\) with \(v\) in \(T\) and \(c(v, w)
		< key(tree(w))\), set \(e(tree(w)) = (v, w)\). If \(key(tree(w)) = \infty\),
		insert \(tree(w)\) in the heap with a redefined key of \(c(v, w)\); if \(c(v,
		w) < key(tree(w)) < \infty\), decrease the key of \(tree(w)\) to \(c(v, w)\).
	\end{step}

	To finish the growth step, empty the heap and set \(key(T) = \infty\) for every
	old tree \(T\) with finite key (these are the trees that have been inserted into
	the heap during the current growth step).
\end{step}

We can analyze the running time of a pass as follows: The time for the cleanup and
other initialization is \bigO{m}. If \(t\) is the number of old trees, the total tile
for growing new trees is \bigO{t \log k + m}: We need at most \(t\) \oper{DeleteMin}
operations, each on a heap of size \(k\) or smaller, and \bigO{m} other heap
operations, none of which is a deletion.

We wish to choose values of \(k\) for successive passes so as to minimize the total
running time. Smaller values of \(k\) reduce the time per pass; larger values reduce
the number of passes. For each pass let use choose \(k = 2^{2m/t}\), where \(m\) is
the original number of edges in the graph and \(t\) is the number of trees before the
pass. The value of \(k\) increases from pass to pass as the number of trees
decreases. With this choice of \(k\), the running time per pass is \bigO{m}.

It remains for us to bound the number of passes. Consider the effect of a pass that
begins with \(t\) trees and \(m' \leq m\) edges (some edges may have been discarded).
Each tree \(T\) remaining after the pass has more than \(k = 2^{2m/t}\) edges with at
least one endpoint in \(T\). (If \(T_0\) is the first old tree among those making up
\(T\) that was placed in the heap, then \(T_0\) was grown until the heap reached size
\(k\), at which time the current tree \(T'\) containing \(T_0\) had more than \(k\)
incident edges. Other trees may have later been connected to \(T'\), causing some of
these incident edges to have \emph{both} their endpoints in the final tree \(T\).)
Since each of the \(m'\) edges has only two endpoints, this means that the number of
trees remaining after the pass, say, \(t'\), satisfies \(t' \leq 2m'/k\). If \(k'\)
is the heap size bound for the next pass, we have \(k' = 2^{2m/t'} \geq 2^k\). Since
the initial heap size bound is \(2m/n\) and a heap size bound of \(n\) or more is
only possible on the last pass, the number of passes is at most \(\min \{ i \mid
\log^{(i)} n \leq 2m/n \} + 1 = \beta(m, n) + \bigO{1}\), where \(\beta(m, n) = \min
\{ i \mid \log^{(i)} n \leq m/n \}\) and \(\log^{(i)} n\) is defined inductively by
\(log^{(0)} n = n, \log^{(i+1)} n = \log \log^{(i)} n\).

Thus we obtain a time bound of \bigO{m \beta(m, n)} for the computation of minimum
spanning trees. This bound improves the old \bigO{m \log \log_{(m/n + 2)} n} bound
for all sufficiently sparse graphs. Note that \(\beta(m, n) \leq log^* n\) if \(m
\geq n\), where \(\log^* n = \min \{ i \mid \log^{(i)} n \leq 1 \}\). (If \(m < n\),
then \(m = n - 1\) and the entire graph is a tree, since we have assumed that the
graph is connected.)

Our fast minimum spanning tree algorithm improves the time bounds for certain kinds
of constrained minimum spanning tree problems as well. For example, one can find
a minimum spanning tree with a degree constraint at one vertex in \bigO{m \beta(m,
n)} time, and a minimum spanning tree with a fixed number of marked edges in a graph
with marked and unmarked edges in \bigO{n \log n + m} time. For details, see Gabow
and Tarjan's paper~\cite{GabowTarjan1984}.

