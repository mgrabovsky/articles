To implement heaps we use heap-ordered trees. A \term{heap-ordered tree} is a rooted
tree containing a set of items, one item in each node, with the items arranged in
\term{heap order}: If \(x\) is any node, then the key of the item in \(x\) is no less
than the key of the item in its parent \(p(x)\), provided \(x\) has a parent. Thus
the tree root contains an item of minimum key. The fundamental operation on
heap-ordered trees is \term{linking}, which combines two item-disjoint trees into
one. Given two trees with roots \(x\) and \(y\), we link them by comparing the keys
of the items in \(x\) and \(y\). If the item in \(x\) has the smaller key, we make
\(y\) a child of \(x\); otherwise, we make \(x\) a child of \(y\). (See
\autoref{fig:linking}.)

\begin{figure}
    \begin{subfigure}[t]{0.50\textwidth}
        \centering
        \begin{tikzpicture}
\node at (0,0) {6}
	child { node {9}
		child { node {16} } }
	child { node {12}
		child { node {20} } }
	child { node {8} };
\node at (3.5,0) {7}
	child { node {10}
		child { node {14}
			child { node {18} } }
		child { node {11} } }
	child { node {40} }
	child { node {15} };
\end{tikzpicture}

        \caption{}
    \end{subfigure}
    \begin{subfigure}[t]{0.49\textwidth}
        \centering
        \begin{tikzpicture}
\node {6}
	child { node {9}
		child { node {16} } }
	child { node {12}
		child { node {20} } }
	child { node {8} }
	child { node {7}
		child { node {10}
			child { node {14}
				child { node {18} } }
			child { node {11} } }
		child { node {40} }
	child { node {15} } };
\end{tikzpicture}


        \caption{}
    \end{subfigure}
    \caption{Linking two heap-ordered trees. (In this and most of the succeeding
        figures, we do not distinguish betweens items and their keys.) (a)~Two trees.
        (b)~After linking.}
    \label{fig:linking}
\end{figure}

A \term{Fibonacci heap} (\term{F-heap}) is a collection of item-disjoint heap-ordered
trees. We impose no explicit constraints on the number or structure of the trees; the
only constraints are implicit in the way the trees are manipulated. We call the
number of children of a node \(x\) its \term{rank} \(r(x)\). There is no constant
upper bound on the rank of a node, although we shall that the rank of a node with
\(n\) descendants is \bigO{\log n}. Each node is either \term{marked} or
\term{unmarked}; we shall discuss the use of marking later.

In order to make the correctness of our claimed time bounds obvious, we assume the
following representation of F-heaps: Each node contains a pointer to its parent (or
to a special node \term{null} if it has no parent) and a pointer to one of its
children. The children of each node are doubly linked in a circular list. Each node
also contains its rank and a bit indicating whether it is marked. The roots of all
the trees in the heap are doubly linked in a circular list. We access the heap by
a pointer to a root containing an item of minimum key; we call this root the
\term{minimum node} of the heap. A minimum node of null denotes the empty heap. (See
\autoref{fig:pointers}.) This representation requires space in each node for one
item, four pointers, an integer, and a bit. The double linking of the lists of roots
and children makes deletion from such a list possible if \bigO{1} time. The circular
linking makes concatenation of lists possible in \bigO{1} time.

\begin{figure}
    \caption{Pointers representing an F-heap. The four pointers in each node indicate
    the left sibling, the parent, some child, and the right sibling. The middle field
    in each node is its key. Ranks and mark bits are not shown.}
    \label{fig:pointers}
\end{figure}

We shall postpone a discussion of \oper{DecreaseKey} and \oper{Delete} until later in
the section. The remaining heap operations we carry out as follows: To perform
\oper{MakeHeap}, we return a pointer to null. To perform \opargs{FindMin}{h}, we
return the item in the minimum node of \(h\). To carry out \opargs{Insert}{i, h}, we
create a new heap consisting of one node containing \(i\), and replace \(h\) by the
meld of \(h\) and the new heap. To carry out \opargs{Meld}{h_1, h_2}, we combine the
root lists of \(h_1\) and \(h_2\) into a single list, and return as the minimum node
of the new heap either the minimum node of \(h_1\) or the minimum node of \(h_2\),
whichever contains the item of the smaller key. (In the case of a tie, the choice is
arbitrary.) All of these operations take \bigO{1} time.

\begin{step}{Linking step}
    Find any two trees whose roots have the same rank, and link them. (The new tree
    root has rank one greater than the ranks of the old tree roots.)
\end{step}

Once there are no two trees with roots of the same rank, we form a list of the
remaining roots, in the process finding a root containing an item of minimum key to
serve as the minimum node of the modified heap. We complete the deletion by saving
the item in \(x\), destroying \(x\), and returning the saved item. (See
\autoref{fig:delete-min}.)

\begin{figure}
    \caption{Effect of a \oper{DeleteMin} operation. (a)~Before the deletion.
    (b)~After removal of the minimum node, containing 3. (c)~After linking 4 and 8,
    then 4 and 6, then 5 and 12.}
    \label{fig:delete-min}
\end{figure}

The \oper{DeleteMin} operation requires finding pairs of tree roots of the same rank
to link. To do this we use an array indexed by rank, from zero up to the maximum
possible rank. Each array position holds a pointer to a tree root. When performing
a \oper{DeleteMin} operation, after deleting the minimum node and forming a list of
the new tree roots, we insert the roots one by one into the appropriate array
positions. Whenever we attempt to insert a root into an already occupied position, we
perform a linking step and attempt to insert the root of the new tree into the next
higher position. After successfully inserting all the roots, we scan the array,
emptying it. The total time for the \oper{DeleteMin} operation is proportional to the
maximum rank of any of the nodes manipulated plus the number of linking steps.

The data structure we have so far described is a ``lazy melding'' version of binomial
queues. If we begin with no heaps and carry out an arbitrary sequence of heap
operations (not including \oper{Delete} or \oper{DecreaseKey}), then each tree ever
created is a \term{binomial tree}, defined inductively as follows: A binomial tree of
rank zero consists of a single node; a binomial tree of rank \(k > 0\) is formed by
linking two binomial trees of rank \(k - 1\). (See \autoref{fig:binomial-trees}.)
A binomial tree of rank \(k\) contains exactly \(2^k\) nodes, and its root has
exactly \(k\) children. Thus every node in a \(n\)-item heap has rank at most \(\log
n\).\footnote{All logarithms in this paper for which a base is not explicitly
specified are base 2.}

\begin{figure}
    \caption{Binomial trees. (a)~Inductive definition. (b)~Examples.}
    \label{fig:binomial-trees}
\end{figure}

We can analyze amortized running times of the heap operations by using the
``potential'' technique of Sleator and Tarjan~\cite{SleatorTarjan1986,Tarjan1985}. We
assign to each possible collection of heaps a real number called the \term{potential}
of the heaps. We define the \term{amortized time} of a heap operation to be its
actual running time plus the net increase it causes in the potential. (A decrease in
potential counts negatively and thus makes the amortized time less than the actual
time.) With this definition, the actual time of a sequence of operations is equal to
the total amortized time plus the net decrease in potential over the entire sequence.

To apply this technique, we define the potential of a collection of heaps to be the
total number of trees they contain. If we begin with no heaps, the initial potential
is zero, and the potential is always nonnegative. Thus the total amortized time of
a sequence of operations is an upper bound on the total actual time. The amortized
time of a \oper{MakeHeap}, \oper{FindMin}, \oper{Insert}, or \oper{Meld} operation is
\bigO{1}: An insertion increases the number of trees by one; the other operations do
not affect the number of trees. If we charge one unit of time for each linking step,
then a \oper{DeleteMin} operation has an amortized time of \bigO{\log n}, where \(n\)
is the number of items in the heap: Deleting the minimum node increases the number of
trees by at most \(\log n\); each linking step decreases the number of trees by one.

Our goal now is to extend this data structure and its analysis to include the
remaining heap operations. We implement \oper{DecreaseKey} and \oper{Delete} as
follows: To carry out \opargs{DecreaseKey}{\Delta, i, h}, we subtract \(\Delta\) from
the key of \(i\), find the node \(x\) containing \(i\), and cut the edge joining
\(x\) to its parent \(p(x)\). This requires removing \(x\) from the list of children
of \(p(x)\) and making the parent pointer of \(x\) null. The effect of the cut is to
make the subtree rooted at \(x\) into a new tree of \(h\), and requires decreasing
the rank of \(p(x)\) and adding \(x\) to the list of roots of \(h\). (See
\autoref{fig:decrease-key}.) (If \(x\) is originally a root, we carry out
\opargs{DecreaseKey}{\Delta, i, h} merely by subtracting \(\Delta\) from the key of
\(i\). If the new key of \(i\) is smaller than the key of the minimum node, we
redefine the minimum node to be \(x\). This method works because \(\Delta\) is
nonnegative; decreasing the key of \(i\) preserves heap order within the subtree
rooted at \(x\), though it may violate heap order between \(x\) and its parent.
A \oper{DecreaseKey} operation takes \bigO{1} actual time.

\begin{figure}
    \caption{The \oper{DecreaseKey} and \oper{Delete} operations. (a)~The original
    heap. (b)~After reducing key 10 to 6. The minimum node is still the node
    containing 3. (c)~After deleting key 7.}
    \label{fig:decrease-key}
\end{figure}

The \oper{Delete} operation is similar to \oper{DecreaseKey}. To carry out
\opargs{Delete}{i, h}, we find the node \(x\) containing \(i\), cut the edge joining
\(x\) and its parent, for a new list of roots by concatenating the list of children
of \(x\) with the original list of roots, and destroy node \(x\). (See
\autoref{fig:decrease-key}.) If \(x\) is originally a root, we remove it from the
list of roots rather than removing it from the list of children of its parent; if
\(x\) is the minimum node of the heap, we proceed as in \oper{DeleteMin}.)
A \oper{Delete} operation takes \bigO{1} actual time, unless the node destroyed is
the minimum node.

There is one additional detail of the implementation that is necessary to obtain the
desired time bounds. After a root node \(x\) has been made a child of another node by
a linking step, as soon as \(x\) loses two of its children through cuts, we cut the
edge joining \(x\) and its parent as well, making \(x\) a new root as in
\oper{DecreaseKey}. We call such a cut a \term{cascading cut}. A single
\oper{DecreaseKey} or \oper{Delete} operation in the middle of a sequence of
operations can cause a possibly large number of cascading cuts. (See
\autoref{fig:cascading-cuts}.)

\begin{figure}
    \caption{Cascading cuts. (a)~Just before decreasing key 10. Nodes 4, 5, and 9 are
    assumed to have previously lost a child via a cut. (b)~After decreasing key 10 to
    7. The original cut separates 10 from 9. Cascading cuts separate 9 from 5, 5 from
    4, and 4 from 2.}
    \label{fig:cascading-cuts}
\end{figure}

The purpose of marking nodes is to keep track of where to make cascading cuts. When
making a root node \(x\) a child of another node in a linking step, we unmark \(x\).
When cutting the edge joining a node \(x\) and its parent \(p(x)\), we decrease the
rank of \(p(x)\) and check whether \(p(x)\) is a root. If \(p(x)\) is not a root, we
mark it if it is unmarked and cut the edge to its parent if it is marked. (The latter
case may lead to further cascading cuts.) With this method, each cut takes \bigO{1}
time.

This completes the description of F-heaps. Our analysis of F-heaps hinges on two
crucial properties: \begin{inparaenum}[(1)]
    \item Each tree in an F-heap, even though not necessarily a binomial tree, has
        a size at least exponential in the rank of its root; and
    \item the number of cascading cuts that take place during a sequence of heap
    operations is bounded by the number of \oper{DecreaseKey} and \oper{Delete}
    operations.
\end{inparaenum}
Before proving these properties, we remark that cascading cuts are introduced in the
manipulation of F-heaps for the purpose of preserving property (1). Moreover, the
condition for their occurrence, namely, the ``loss of two children'' rule, limits the
frequency of cascading cuts as described by property (2). The following lemma implies
property (1).

\begin{lemma}
    \label{lemma:child-rank}
    Let \(x\) be any node in an F-heap. Arrange the children of \(x\) in the order
    they were linked to \(x\), from earliest to latest. Then the \(i\)th child of
    \(x\) has a rank of at least \(i - 2\).
\end{lemma}

\begin{proof}
    Let \(y\) be the \(i\)th child of \(x\), and consider the time when \(y\) was
    linked to \(x\). Just before the linking, \(x\) had at least \(i - 1\) children
    (some of which it may have lost after the linking). Since \(x\) and \(y\) had the
    same rank just before the linking, they both had a rank of at least \(i - 1\) at
    this time. After the linking, the rank of \(y\) could have decreased by at most
    one without causing \(y\) to be cut as a child of \(x\).
\end{proof}

\begin{corollary}
    \label{cor:golden-node}
    A node of rank \(k\) in an F-heap has at least \(F_{k+2} \geq \phi^k\)
    descendants, including itself, where \(F_k\) is the \(k\)th Fibonacci number
    (\(F_0 = 0, F_1 = 1, F_k = F_{k-2} + F_{k-1}\) for \(k \geq 2\)), and \(\phi = (1
    + \sqrt 5) / 2\) is the golden ratio. (See \autoref{fig:fibonacci}.)
\end{corollary}

\begin{proof}
    Let \(S_k\) be the minimum possible number of descendants of a node of rank
    \(k\). Obviously, \(S_0 = 1\), and \(S_1 = 2\). \autoref{lemma:child-rank}
    implies that \(S_k \geq \sum_{i=0}^{k-2} S_i + 2\) for \(k \geq 2\). The
    Fibonacci numbers satisfy \(F_{k+2} = \sum_{i=2}^k F_i + 2\) for \(k \geq 2\),
    from which \(S_k \geq F_{k+2}\) for \(k \geq 0\) follows by induction on \(k\).
    The inequality \(F_{k+2} \geq \phi^k\) is well known~\cite{TAOCP1}.
\end{proof}

\begin{remark}
    This corollary is the source of the name ``Fibonacci heap.''
\end{remark}

\begin{figure}
    \caption{Trees of minimum possible size for a given rank in a Fibonacci heap.}
    \label{fig:fibonacci}
\end{figure}

To analyze F-heaps we need to extend our definition of potential. We define the
potential of a collection of F-heaps  to be the number of trees plus twice the number
of marked nonroot nodes. The \bigO{1} amortized time bounds for \oper{MakeHeap},
\oper{FindMin}, \oper{Insert}, and \oper{Meld} remaind valid, as does the \bigO{\log
n} bound for \oper{DeleteMin}; \opargs{DeleteMin}{h} increases the potential by at
most \(1.4404 \log n\) minus the number of linking steps, since, if the minimum node
has rank \(k\), then \(\phi^k \leq n\) and thus \(k \leq \log n\,/\log \phi \leq
1.4404 \log n\).

Let us charge one unit of time for each cut. A \oper{DecreaseKey} operation causes
the potential to increase by at most three minus the number of cascading cuts, since
the first cut converts a possibly unmarked nonroot node into a root, each cascading
cut converts a marked nonroot node into a root, and the last cut (either first or
cascading) can convert a nonroot node from unmarked to marked. It follows that
\oper{DecreaseKey} has an \bigO{1} amortized time bound. Combining the analysis of
\oper{DecreaseKey} with that of \oper{DeleteMin}, we obtain and \bigO{\log n}
amortized time bound for \oper{Delete}. Thus we have the following theorem.

\begin{theorem}
    If we begin with no F-heaps and perform an arbitrary sequence of F-heap
    operations, then the total time is at most the total amortized time, where the
    amortized time is \bigO{\log n} for each \oper{DeleteMin} or \oper{Delete}
    operation and \bigO{1} for each of the other operations.
\end{theorem}

We close this section with a few remarks on the storage utilization of F-heaps. Out
implementation of F-heaps uses four pointers per node, but this can be reduced to
three per node by using an appropriate alternative representation~\cite{Brown1978}
and even to two per node by using a more complicated representation, at a cost of
a constant factor in running time. Although our implementation uses an array for
finding roots of the same rank to link, random-access memory is not actually
necessary for this purpose.  Instead, we can maintain a doubly linked list of
\term{rank nodes} representing the possible ranks. Each node has a \term{rank
pointer} to the rank node representing its rank. Since the rank of a node is
initially zero and only increases or decreases by one, it is easy to maintain the
rank pointers. When we need to carry out linking steps, we can use each rank node to
hold a pointer to a root of the appropriate rank.  Thus the entire data structure can
be implemented on a pointer machine~\cite{Tarjan1979a} with no loss in asymptotic
efficiency.

% vim: set et sw=4:
